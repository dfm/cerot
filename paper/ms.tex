% Copyright 2017-2018 Dan Foreman-Mackey and the co-authors listed below.

\documentclass[modern]{aastex61}

\usepackage{microtype}
\usepackage{url}
\usepackage{amsmath}
\usepackage{amssymb}
\usepackage{natbib}
\usepackage{multirow}
\bibliographystyle{aasjournal}

% ------------------ %
% end of AASTeX mods %
% ------------------ %

% Projects:
\newcommand{\project}[1]{\textsf{#1}}
\newcommand{\kepler}{\project{Kepler}}
\newcommand{\ktwo}{\project{K2}}
\newcommand{\lsst}{\project{LSST}}
\newcommand{\tess}{\project{TESS}}
\newcommand{\celerite}{\project{celerite}}
\newcommand{\emcee}{\project{emcee}}

\newcommand{\foreign}[1]{\emph{#1}}
\newcommand{\etal}{\foreign{et\,al.}}
\newcommand{\etc}{\foreign{etc.}}
\newcommand{\ie}{\foreign{i.e.}}

\newcommand{\figureref}[1]{\ref{fig:#1}}
\newcommand{\Figure}[1]{Figure~\figureref{#1}}
\newcommand{\figurelabel}[1]{\label{fig:#1}}

\newcommand{\Table}[1]{Table~\ref{tab:#1}}
\newcommand{\tablelabel}[1]{\label{tab:#1}}

\renewcommand{\eqref}[1]{\ref{eq:#1}}
\newcommand{\Eq}[1]{Equation~(\eqref{#1})}
\newcommand{\eq}[1]{\Eq{#1}}
\newcommand{\eqalt}[1]{Equation~\eqref{#1}}
\newcommand{\eqlabel}[1]{\label{eq:#1}}

\newcommand{\sectionname}{Section}
\newcommand{\sectref}[1]{\ref{sect:#1}}
\newcommand{\Sect}[1]{\sectionname~\sectref{#1}}
\newcommand{\sect}[1]{\Sect{#1}}
\newcommand{\sectalt}[1]{\sectref{#1}}
\newcommand{\App}[1]{Appendix~\sectref{#1}}
\newcommand{\app}[1]{\App{#1}}
\newcommand{\sectlabel}[1]{\label{sect:#1}}

\newcommand{\T}{\ensuremath{\mathrm{T}}}
\newcommand{\dd}{\ensuremath{\,\mathrm{d}}}
\newcommand{\unit}[1]{{\ensuremath{\,\mathrm{#1}}}}
\newcommand{\bvec}[1]{{\ensuremath{\boldsymbol{#1}}}}

% TO DOS
\newcommand{\todo}[3]{{\color{#2}\emph{#1}: #3}}
\newcommand{\dfmtodo}[1]{\todo{DFM}{red}{#1}}

% \shorttitle{}
% \shortauthors{}
% \submitted{Submitted to \textit{The AAS Journals}}

% typography obsessions
\setlength{\parindent}{3.0ex}

\begin{document}\raggedbottom\sloppy\sloppypar\frenchspacing

\title{%
Inferring stellar rotation periods using K2 at scale
}

\author[0000-0002-9328-5652]{Daniel Foreman-Mackey}
\affil{Center for Computational Astrophysics, Flatiron Institute, New York, NY}

\author[0000-0003-4540-5661]{Ruth Angus}
\affil{Simons Fellow}
\affil{Department of Astronomy, Columbia University, New York, NY}

\author[0000-0002-0296-3826]{Rodrigo Luger}
\affil{Department~of~Astronomy, University~of~Washington, Seattle, WA}

\begin{abstract}

We measure all the periods.

\end{abstract}

\keywords{%
 %methods: data analysis
 %---
 %methods: statistical
 %---
 %asteroseismology
 %---
 %stars: rotation
 %---
 %planetary systems
}

\section{Introduction}

Some words\ldots \citep{Luger:2017}

\citep{Foreman-Mackey:2017} demonstrated that \celerite\ could be used to
compute the likelihood of a GP model where the kernel is a mixture of
stochastically-driven, damped simple harmonic oscillators (SHOs).
The power spectral density of each term in this model is:
\begin{eqnarray}
S_k(\omega) = \sqrt{\frac{2}{\pi}}\,
\frac{s_k\,\omega_k}{{(\omega^2 - {\omega_k}^2)}^2 +
    {\omega_k}^2\,\omega^2/{q_k}^2}
\end{eqnarray}
where $s_k$ is the driving power, $\omega_k$ is the un-damped frequency, and
$q_k$ is the quality factor of the oscillator.
We find that a restricted mixture of three SHO terms is flexible enough to
capture the astrophysical variability while remaining simple enough to be
easily interpreted.
We set the quality factor of the first oscillator to $Q_1 = 1/\sqrt{2}$ to
capture the stellar granulation using a ``Harvey model''.
We constrain the other two terms with
\begin{eqnarray}
a_2 &\ge& a_3, \\
P_2 &=& 2\,P_3,\quad \mathrm{and} \\
q_2 &\ge& q_3 > 1/2
\end{eqnarray}
where $a_k = s_k\,\omega_k\,q_k$ is the amplitude of the oscillator and
$P_k = 4\,\pi\,q_k / \sqrt{4\,{q_k}^2-1}\,\omega_k$ is the oscillation period.



\acknowledgments\
It is a pleasure to thank
\dfmtodo{many people}
for helpful discussions informing the ideas and code presented here.

This research made use of the NASA \project{Astrophysics Data System} and the
NASA Exoplanet Archive.
The Exoplanet Archive is operated by the California Institute of Technology,
under contract with NASA under the Exoplanet Exploration Program.

This paper includes data collected by the \kepler\ Mission. Funding for the
\kepler\ Mission is provided by the NASA Science Mission directorate.
We are grateful to the entire \kepler\ team, past and present.
These data were obtained from the Mikulski Archive for Space Telescopes
(MAST).
STScI is operated by the Association of Universities for Research in
Astronomy, Inc., under NASA contract NAS5\-26555.
Support for MAST is provided by the NASA Office of Space Science via grant
NNX13AC07G and by other grants and contracts.

%This research made use of Astropy, a community-developed core Python package
%for Astronomy \citep{Astropy-Collaboration:2013}.

\facility{Kepler}
\software{%
     %\project{AstroPy} \citep{Astropy-Collaboration:2013},
     \project{corner.py} \citep{Foreman-Mackey:2016},
     \project{Eigen} \citep{Guennebaud:2010},
     \project{emcee} \citep{Foreman-Mackey:2013},
     %\project{george} \citep{Ambikasaran:2016},
     %\project{Julia} \citep{Bezanzon:2012},
     %\project{LAPACK} \citep{Anderson:1999},
     \project{matplotlib} \citep{Hunter:2007},
     \project{numpy} \citep{Van-Der-Walt:2011},
     %\project{transit} \citep{Foreman-Mackey:2016a},
     \project{scipy} \citep{Jones:2001}.
}

%\vspace{5ex}
\appendix

There's always an appendix.

\bibliography{rotate}

\end{document}
